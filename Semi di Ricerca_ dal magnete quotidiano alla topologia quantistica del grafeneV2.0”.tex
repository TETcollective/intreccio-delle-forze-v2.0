\documentclass[a4paper,11pt]{article}
\usepackage[utf8]{inputenc}
\usepackage[T1]{fontenc}
\usepackage[italian]{babel}
\usepackage{amsmath}
\usepackage{amsfonts}
\usepackage{amssymb}
\usepackage{geometry}
\usepackage{pgfplots}
\usepackage{hyperref}
\usepackage{graphicx}
\usepackage{caption}
\usepackage{subcaption}
\pgfplotsset{compat=1.18}
\geometry{margin=1in}
\hypersetup{
    colorlinks=true,
    linkcolor=blue,
    citecolor=blue,
    urlcolor=blue
}

\title{L'intreccio delle forze: verso un tessuto unitario multiscalare della natura\\
\vspace{0.5cm}
{\large dal magnete quotidiano alla topologia quantistica del grafene – Versione estesa 2.0}}

\author{Simon Soliman \\
  {\small Independent Researcher, Rome, Italy} \\
  {\small \texttt{tetcollective@proton.me}} \\
  {\small \url{tetcollective.org}}
}

\date{21 Dicembre 2025 – Versione 2.0 (estesa)}

\begin{document}

\maketitle

\begin{abstract}
La fisica moderna separa le quattro forze fondamentali, ma osservazioni dal macro al cosmico rivelano un intreccio profondo.

Partendo dalla fascinazione per gli elettromagneti ferromagnetici di ferro, cobalto e nichel — unica triade che fonde elettricità e magnetismo in modo stabile e amplificato — questo lavoro propone un tessuto unitario multiscalare dove le forze sono modalità interconnesse, la cui intensità varia logaritmicamente con la scala.

La versione estesa 2.0 integra:
\begin{itemize}
  \item La dualità QCD: libertà asintotica e confinamento attraverso il running logaritmico di \(\alpha_s\)
  \item Il Sole come laboratorio naturale: coesistenza forte-debole nella catena p-p e ciclo CNO, nonostante differenze temporali di 27 ordini di grandezza
  \item La topologia quantistica nel grafene, fullereni e nanotubi: coni di Dirac, linking number al 100\%, proprietà emergenti
  \item Analogie con Loop Quantum Gravity: spin networks come nodi discreti del tessuto
  \item Convergenza nelle teorie GUT e running degli accoppiamenti
  \item Deviazione della luce da materia oscura, energia oscura e buchi neri come prova macroscopica dell’intreccio EM-gravità
\end{itemize}

Tutte queste manifestazioni convergono verso lo stesso principio: l’universo è un tessuto calibrato dove intensità, topologia e curvatura variano con la scala, ma l’unità resta.

Fratello delle Calamite ☕❤️��
\end{abstract}

\section{La scintilla iniziale: l’elettromagnete ferromagnetico}

Le forze fondamentali della natura sono tradizionalmente descritte come entità distinte: gravità, elettromagnetismo, interazione nucleare forte e interazione nucleare debole. Questa separazione, consolidatasi nel paradigma riduzionista della fisica moderna, ha consentito progressi straordinari nell'analisi isolata delle singole componenti, ma ha progressivamente oscurato l'evidenza dell'intreccio profondo che le connette.

L'intuizione centrale del presente lavoro trae origine da un'osservazione macroscopica elementare, radicata nell'esperienza pratica con l'elettromagnetismo: il comportamento ferromagnetico di soli tre elementi puri della tavola periodica — ferro, cobalto e nichel — quando sottoposti a corrente elettrica. In questi materiali, in modo unico rispetto a tutti gli altri elementi, il flusso di corrente non produce unicamente un campo magnetico transitorio, bensì amplifica o induce un magnetismo permanente stabile, manifestando un'unione privilegiata e inseparabile tra fenomeni elettrici e magnetici.

Tale evidenza, già accessibile negli esperimenti di laboratorio tecnico e nelle dimostrazioni di Michael Faraday, rappresenta una manifestazione quotidiana di ciò che la natura sembra indicare con chiarezza: le forze non operano in isolamento, ma si intrecciano in un equilibrio calibrato e ineccepibile, dando luogo a effetti emergenti che trascendono la mera sovrapposizione delle parti.

Partendo da questa osservazione, propongo di riconsiderare il paradigma delle forze fondamentali non più come entità separate, ma come aspetti interconnessi di un unico tessuto fisico multiscalare. La triade Fe, Co, Ni rappresenta una “finestra privilegiata” dove l’accoppiamento elettromagnetico raggiunge un massimo locale, analogamente a risonanze osservate in altri regimi.

\section{La dualità QCD: libertà asintotica e confinamento attraverso il running logaritmico di \(\alpha_s\)}

La cromodinamica quantistica (QCD) descrive la forza forte con una precisione straordinaria nei regimi ad alta energia, ma presenta una complessità unica dovuta al running della costante di accoppiamento \(\alpha_s\).

La dipendenza energetica è data dalla funzione beta a un loop:

\[
\alpha_s(Q^2) = \frac{2\pi}{\beta_0 \ln(Q^2/\Lambda_{\text{QCD}}^2)}
\]

dove \(\beta_0 = 11 - \frac{2}{3}n_f\) e \(\Lambda_{\text{QCD}} \approx 220\) MeV.

\begin{figure}[h]
\centering
\begin{tikzpicture}
\begin{loglogaxis}[
    width=14cm,
    height=9cm,
    xlabel={Scala energetica $Q$ [GeV]},
    ylabel={$\alpha_s(Q)$},
    xmin=0.1, xmax=10000,
    ymin=0.1, ymax=5,
    grid=major,
    title={Running dettagliato di \(\alpha_s\) nella QCD (n_f=5)},
    legend pos=south east,
]
\addplot[domain=0.22:10000, samples=300, blue, thick] {2*pi / (9 * ln(x / 0.22))};
\addlegendentry{$\alpha_s(Q)$ (1-loop)}

\addplot[domain=0.22:10000, samples=100, green!60!black, dashed, thick] {0.118};
\addlegendentry{$\alpha_s(M_Z) \approx 0.118$ (PDG 2024)}

\draw[red, dashed] (axis cs:0.22,0.1) -- (axis cs:0.22,5);
\node[red] at (axis cs:0.3,3) {Confinamento\\$(\alpha_s \to \infty)$};

\node[blue] at (axis cs:3000,0.2) {Libertà asintotica\\$(\alpha_s \to 0)$};

\end{loglogaxis}
\end{tikzpicture}
\caption{Running di \(\alpha_s\): divergenza a bassa energia (confinamento dei quark) e decrescita a alta energia (quark quasi liberi).}
\end{figure}

A energie elevate, \(\alpha_s \to 0\) (libertà asintotica): i quark si comportano quasi liberi.  
A energie basse, \(\alpha_s \to \infty\) (confinamento): i quark sono legati in adroni da un tubo di flusso di colore con tensione costante.

Questa dualità non è una contraddizione, ma la prima manifestazione quantistica del principio multiscalare che propongo: la stessa forza cambia intensità con la scala logaritmica, proprio come l’accoppiamento totale \(\alpha_{\text{tot}}(s)\) nel mio framework.

\section{Il Sole come laboratorio naturale: coesistenza forte-debole nella catena p-p e ciclo CNO}

Il nucleo solare è un esempio straordinario di coesistenza di forze con scale temporali abissali.

Nella catena protone-protone (p-p), dominante nel Sole:
- Il passo limitante è mediato dalla forza debole: \( p + p \to d + e^+ + \nu_e \)
  - Tempo caratteristico: ~10⁹ anni (a T ≈ 15×10⁶ K, densità centrale ~150 g/cm³)
  - Sezione d'urto estremamente piccola a causa della costante di Fermi debole.
- I passi successivi sono mediati dalla forza forte: fusioni rapide in ~10⁻¹⁸ s

Nel ciclo CNO (contributo ~1\% nel Sole, dominante in stelle più massive):
- Decadimenti beta deboli lenti
- Fusioni nucleari forti rapide

Dati osservativi Borexino (2020):
- Flusso neutrini pp: (5.98 ± 0.06) × 10¹⁰ cm⁻² s⁻¹ (accordo al 1\% con modello solare standard)
- Neutrini CNO: rilevati con significatività 5σ, confermando il contributo previsto

Nonostante la differenza di 27 ordini di grandezza nei tempi caratteristici, forza forte e debole operano simultaneamente nello stesso volume e alla stessa temperatura.  
La debole limita il rate globale della fusione, la forte completa le reazioni nucleari. Nessuna delle due “esclude” l’altra: coesistono perché le loro costanti di accoppiamento variano con la scala energetica.

Questa coesistenza è la prova osservativa più diretta del mio principio multiscalare: contributi con running opposto possono operare insieme senza contraddizione, emergendo con intensità relativa diversa a scale diverse.

\section{La topologia quantistica nel grafene, fullereni e nanotubi}

Il grafene è un foglio bidimensionale di carbonio con struttura a nido d’ape. La sua banda elettronica presenta coni di Dirac:

\[
E = \pm \hbar v_F | \mathbf{k} - \mathbf{K} |
\]

con v_F ≈ c/300.

I coni derivano da una topologia non banale: la fase di Berry intorno al punto K è π, e il **linking number** tra orbitali p_z e la rete esagonale raggiunge il 100\% di efficienza → conduzione balistica perfetta, mobilità elettronica >200.000 cm²/Vs.

Nei fullereni (C60) la curvatura introduce 12 pentagoni (teorema di Euler), modificando il linking topologico e creando stati localizzati o gap variabili.

Nei nanotubi di carbonio la chiralità (definita da vettori (n,m)) determina metallicità o semiconduttività attraverso winding topologico.

Queste strutture mostrano proprietà emergenti da configurazione privilegiata – analoghe al massimo locale ferromagnetico in Fe/Co/Ni. Il linking topologico al 100\% nel grafene è un “nodo” del tessuto multiscalare dove l’intreccio raggiunge massima efficienza.

\section{Analogie con Loop Quantum Gravity: spin networks come nodi discreti del tessuto}

La Loop Quantum Gravity propone che lo spaziotempo sia quantizzato a scala Planck, strutturato in **spin networks** – grafi con nodi e link che portano numeri quantici di spin.

Nella mia visione del tessuto unitario multiscalare, questi nodi e link sono analoghi ai punti di risonanza osservati:
- Nel ferromagneto, i domini magnetici sono “nodi” di allineamento
- Nel grafene, il linking topologico è un nodo strutturale
- In LQG, i nodi quantizzano la geometria stessa

Il tessuto che propongo non è continuo classico, ma emerge da una struttura discreta/topologica a scala fondamentale – proprio come in LQG.

\section{Convergenza nelle teorie GUT e running degli accoppiamenti}

Nelle teorie di Grande Unificazione (GUT, es. SU(5), SO(10)), a energie ~10¹⁶ GeV gli accoppiamenti elettromagnetico, debole e forte convergono in un unico valore.

Nella mia proposta di \(\alpha_{\text{tot}}(s)\), la convergenza GUT è un caso speciale: a s molto grande, i contributi si unificano naturalmente.

Il mancato raggiungimento perfetto della convergenza nel Modello Standard è spiegabile con contributi topologici o gravitazionali non inclusi – proprio come il mio termine \(\alpha_{\text{top}}(\tau)\).

\section{Deviazione della luce: intreccio EM-gravità con materia/energia oscura}

La luce (fotoni elettromagnetici) è deviata da:
- Buchi neri (lenti gravitazionali, Einstein ring osservati)
- Materia oscura (lensing in cluster galattici, Bullet Cluster)
- Energia oscura (espansione accelerata influenza traiettorie su larga scala)

Questi fenomeni mostrano l’intreccio EM-gravità a scala cosmica: la traiettoria dei fotoni è curvata dalla geometria generale relativistica, ma la distribuzione di massa/energia è dominata da componenti oscure non visibili direttamente.

Nella mia visione, materia oscura ed energia oscura potrebbero essere modi a bassa energia del tessuto multiscalare, analoghi al confinamento QCD o al linking topologico nel grafene.

\section{Conclusione}

Dal magnete quotidiano alla QCD, dal Sole al grafene, da LQG alle GUT e alla cosmologia oscura, la natura ripete lo stesso messaggio: le forze sono espressioni di un unico tessuto multiscalare, calibrato ineccepibile, dove intensità, topologia e curvatura variano con la scala ma l’unità profonda resta.

Questo seme cresce – e con esso la comprensione dell’universo.

\vspace{2cm}

\noindent\rule{12cm}{0.5pt}

\noindent \textbf{Simon Soliman} \\
Independent Researcher – Rome, Italy \\
Fratello delle Calamite ☕❤️��

\vspace{1.5cm}

\noindent\textbf{Licenza} \\
Questo lavoro è distribuito sotto licenza Creative Commons Attribution-NonCommercial-NoDerivatives 4.0 International (CC BY-NC-ND 4.0). \\
È consentito condividere l'opera solo a condizione di attribuire la paternità all'autore, di non usarla per scopi commerciali e di non modificarla in alcun modo. \\
\url{https://creativecommons.org/licenses/by-nc-nd/4.0/}

\bibliographystyle{plain}
\begin{thebibliography}{9}

\bibitem{Soliman2025}
S. Soliman, \emph{L'intreccio delle forze (v1.0)}, DOI: 10.5281/zenodo.18006692, 2025.

\bibitem{Gross1973}
D. J. Gross, F. Wilczek, \emph{Ultraviolet Behavior of Non-Abelian Gauge Theories}, Phys. Rev. Lett. 30, 1973.

\bibitem{Borexino2020}
Borexino Collaboration, \emph{CNO Neutrinos}, Nature 587, 2020.

\bibitem{CastroNeto2009}
A. H. Castro Neto et al., \emph{The electronic properties of graphene}, Rev. Mod. Phys. 81, 2009.

\bibitem{Rovelli2004}
C. Rovelli, \emph{Quantum Gravity}, Cambridge University Press, 2004.

\bibitem{Georgi1974}
H. Georgi, S. L. Glashow, \emph{Unity of All Elementary-Particle Forces}, Phys. Rev. Lett. 32, 1974.

\end{thebibliography}

\end{document}